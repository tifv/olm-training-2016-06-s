% $date: 2016-06-26
% $timetable:
%   g10r3:
%     2016-06-26:
%       1:

\section*{Линейные рекурренты}

% $authors:
% - Аскар Флоридович Назмутдинов

\definition
Последовательность чисел $a_{0}$, $a_{1}$, \ldots, $a_{n}$, \ldots, которая
удовлетворяет с~заданными $p$ и~$q$ соотношению
\[
    a_{n+2}
=
    p a_{n+1} + q a_{n}
\, , \quad
    n = 0, 1, 2, \ldots
\]
называется
\emph{линейной рекуррентной (возвратной) последовательностью второго порядка.}

Уравнение
\[
    x^2 - p x - q = 0
\]
называется \emph{характеристическим} уравнением
последовательности $\{ a_{n}\}$.

\begin{problems}

\itemy{0}
Пусть последовательности $a_{n}$ и~$b_{n}$ являются линейными рекуррентными
последовательностями второй порядка, с~одинаковыми $p$ и $q$.
Докажите, что последовательность $\alpha a_{n} + \beta b_{n}$ также является
реккурентной последовательностью второго порядка (с~теми~же $p$, $q$).

\item
\subproblem
Докажите, что геометрическая прогрессия $\{ a_{n} \} = b x_0^{n}$ удовлетворяет
соотношению $a_{n+2} = p a_{n+1} + q a_{n}$ ($n = 0, 1, 2, \ldots$)
тогда и~только тогда, когда $x_0$~--- корень характеристического уравнения
$x^2 - p x - q = 0$.
\\
\subproblem
Пусть характеристическое уравнение последовательности $\{ a_{n} \}$ имеет два
различных корня $x_1$ и~$x_2$.
Докажите, что при фиксированных $a_{0}, a_{1}$ существует ровно одна пара чисел
$c_1$, $c_2$ такая, что $a_{n} = c_1 x_1^{n} + c_2 x_2^{n}$ ($n \geq 0$).

\item
Найдите формулу $n$-го члена для последовательностей, заданных условиями:
\\
\subproblem
$a_{0} = 0$, $a_{1} = 1$, $a_{n+2} = 5 a_{n+1} - 6 a_{n}$, $n \geq 0$;
\\
\subproblem
$a_{0} = 1$, $a_{1} = 2$, $a_{n+2} = 2 a_{n+1} - a_{n}$, $n \geq 0$;
\\
\subproblem \emph{Числа Фибоначчи.}\enspace
$F_{0} = 0, F_{1} = 1, F_{n+2} = F_{n+1} + F_{n}$, $n \geq 0$.

\item
Пусть характеристическое уравнение последовательности $\{ a_{n} \}$ имеет
корень $x_0$ кратности $2$.
Докажите, что при фиксированных $a_{0}$, $a_{1}$ существует ровно одна пара
чисел $c_1, c_2$ такая, что
\[
    a_{n} = (c_1 + c_2 n) x_0^{n}
\, , \quad
    n = 0, 1, 2, \ldots
\, . \]
\item
Садовник, привив черенок редкого растения, оставляет его расти два года,
а~затем ежегодно берет от~него по~$6$ черенков.
С~каждым новым черенком он поступает аналогично.
Сколько будет растений и~черенков на~$n$-ом году роста первоначального
растения?

\item
Биолог выращивает микробов, живущих по~следующему принципу:
в~первый день после рождения, ровно в~8:30 каждый микроб порождает
5 новых микробов.
во~второй и~последующие дни после рождения ровно в~9:00 каждый съедает
4 (новорожденных) микроба.
Изначально у~биолога был 1~микроб, сколько микробов будет у~него
на~$n$-ый день?

\item
Лягушка прыгает по~вершинам треугольника $ABC$, перемещаясь каждый раз в~одну
из~соседних вершин.
Сколькими способами она может попасть из~$A$ в~$A$ за~$n$ прыжков?

\item
В~нулевой момент времени в~вершине~$A$ шестиугольника $ABCDEF$ сидит лягушка.
Каждую секунду лягушка перепрыгивает в~одну из~соседних вершин, выбирая
направление случайным образом равновероятно.
Сколькими способами она может попасть из~$A$ в~$C$ за~$n$ прыжков?

\end{problems}


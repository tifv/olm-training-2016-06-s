% $date: 2016-06-24
% $timetable:
%   g10r3:
%     2016-06-24:
%       2:

\section*{Показатели}

% $authors:
% - Аскар Флоридович Назмутдинов

\begingroup
    \ifdefined\mathup
        \def\eulerphi{\mathup{\phi}}%
    \else
        \def\eulerphi{\upphi}%
    \fi
    \def\divides{\mathrel{\vert}}%

\begin{problems}

\item
Пусть $a$, $n$~--- взаимно простые числа.
Рассмотрим последовательность остатков по~модулю~$n$ следующих чисел:
$1$, $a$, $a^{2}$, \ldots
Докажите, что эта последовательность периодическая и~не~содержит предпериода.

\end{problems}

\definition
Минимальный период последовательности остатков из~предыдущей задачи называется
\emph{показателем} $a$ по~модулю $n$.
Далее будем обозначать его буквой $d$.

\begin{problems}

\item
Зафиксируем взаимно простые числа $a$ и~$n$.
\\
\subproblem
Докажите, что $d$~--- показатель $a$ по~модулю~$n$ тогда и~только тогда, когда
$d$~--- наименьшее такое натуральное число, что $(a^{d} - 1)$ делится на~$n$.
\\
\subproblem
Пусть $d$~--- показатель $a$ по~модулю~$n$.
Пусть $a^{l} \equiv 1 \pmod{n}$.
Докажите, что $d \divides l$.
\\
\subproblem
Докажите, что $a^{s} \equiv a^{r} \pmod{n}$ тогда и~только тогда, когда
$s \equiv r \pmod{d}$.
\\
\subproblem
Докажите, что показатель любого взаимно простого с~$n$ числа по~модулю~$n$
делит $\eulerphi(n)$ (функция Эйлера).

\item
Найдите все простые $p$ и~$q$ такие, что $q \divides (2^{p} - 1)$
и~$p \divides (2^{q} - 1)$.

\item
Докажите, что если $a > 1$, то~$n$ делит $\eulerphi(a^{n} - 1)$.

\item
\subproblem
Пусть $p > 2$~--- простое число.
Докажите, что любой простой делитель числа $(a^{p} - 1)$ или делит $(a - 1)$
или имеет вид $2 p x + 1$.
\\
\subproblem
Выведите отсюда, что простых чисел вида $2 p k + 1$ бесконечно много.

\item
Найдите все простые $p$ и~$q$, для которых $5^{p} + 5^{q}$ делится на~$p q$.

\item
Найдите все натуральные~$n$ такие, что $n \divides (2^{n} - 1)$.

\end{problems}

\endgroup % \def\eulerphi \def\divides


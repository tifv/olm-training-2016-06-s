% $date: 2016-06-22
% $timetable:
%   g10r3:
%     2016-06-22:
%       1:

\section*{Теоремы Ферма и Эйлера}

% $authors:
% - Аскар Флоридович Назмутдинов

\begingroup
    \ifdefined\mathup
        \def\eulerphi{\mathup{\phi}}%
    \else
        \def\eulerphi{\upphi}%
    \fi

\claim{Малая теорема Ферма (1)}
Пусть $p$~--- простое число, $a$ не~делится на~$p$.
Тогда $(a^{p-1} - 1)$ делится на~$p$.

\claim{Малая теорема Ферма (2)}
Пусть $a$~--- целое число, $p$~--- простое.
Тогда $(a^{p} - a)$ делится на~$p$.

\claim{Теорема Эйлера}
Пусть $a$ и~$n$~--- натуральные, взаимно простые числа.
Определим \emph{функцию Эйлера} $\eulerphi(n)$ как количество натуральных
чисел, не~превосходящих $n$ и~взаимно простых с~ним.
Тогда $(a^{\eulerphi(n)} - 1)$ делится на~$n$.

\begin{problems}

\item
Сумма трех чисел $a$, $b$, $c$ делится на~$30$.
Докажите, что $a^{5} + b^{5} + c^{5}$ также делится на~$30$.

\item
Найдите все такие целые числа~$a$, для которых число $a^{10} + 1$ делится
на~$10$.

\item
Докажите, что $30^{239} + 239^{30}$~--- составное.

\item
Существует~ли степень тройки, заканчивающаяся на~$0001$?

\item
\subproblem
Докажите, что ни~при каком целом~$k$ число $k^{2} + k + 1$ не~делится на~$101$.
\\
\subproblem
Пусть $p$~--- простое число и~$p > 3$.
Докажите, что если разрешимо сравнение $x^{2} + x + 1 \equiv 0 \pmod{p}$,
то~$p \equiv 1 \pmod{6}$ .
Выведите отсюда бесконечность множества простых чисел вида $6 n + 1$.
\\
\subproblem
Пусть $p$~--- простое число и~$p > 5$.
Докажите, что если разрешимо сравнение
$x^{4} + x^{3} + x^{2} + x + 1 \equiv 0 \pmod{p}$,
то~$p \equiv 1 \pmod{5}$ .
Выведите отсюда бесконечность множества простых чисел вида $5 n + 1$.

\item \emph{Теорема Вильсона.}
Пусть $p$~--- простое число.
Докажите, что $(p - 1)! \equiv -1 \pmod{p}$.

\item
\subproblem
Пусть $p$, $q$~--- различные натуральные числа, $a$~--- натуральное.
Найдите $\eulerphi(p^{a})$, $\eulerphi(pq)$.
\\
\subproblem
Докажите, что если $a$, $b$~--- взаимно просты,
то~$\eulerphi(a b) = \eulerphi(a) \cdot \eulerphi(b)$
\\
\subproblem
Найдите $\eulerphi(n)$, где
$n = p_{1}^{\alpha_{1}} \cdot \ldots \cdot p_{k}^{\alpha_{k}}$.

\item
\emph{Усиление теоремы Эйлера.}
Пусть $m = p_{1}^{\alpha_{1}} \cdot \ldots \cdot p_{k}^{\alpha_{k}}$,
\(
    x
=
    \text{HOK} \bigl(
        \eulerphi(p_{1}^{\alpha_{1}}),
        \ldots,
        \eulerphi(p_{k}^{\alpha_{k}})
    \bigr)
\).
Докажите, что для любого $a$, взаимно простого с~$m$, выполняется сравнение
$a^{x} \equiv 1 \pmod{m}$.

\item
Пусть $p$~--- простое число, тогда
$(11{\ldots}122{\ldots}233{\ldots}99 - 123456789)$
(в~первом числе каждая цифра встречается ровно $p$~раз) делится на~$p$.

\item
Докажите, что в~любой арифметической прогрессии составленной из~натуральных
чисел, есть бесконечно много членов, в~разложении которых на~простые множители
входят в~точности одни и~те~же простые числа.

\end{problems}

\endgroup % \def\eulerphi


% $date: 2016-06-25
% $timetable:
%   g10r1:
%     2016-06-25:
%       2:

\section*{Разнобой по теории чисел}

% $authors:
% - Леонид Андреевич Попов

\begingroup
    \def\divides{\mathrel{\vert}}%

\begin{problems}

\item
Найдите наибольший общий делитель чисел
\[
    A_{n} = 2^{3n} + 3^{6n+2} + 5^{6n+2}
\, , \quad
    \text{где $n = 0, 1, \ldots, 2016$}
.\]

\item
Многочлен $f(x)$ с~целыми коэффициентами называется \emph{три-делимым}, если
$3$ делит $f(k)$ для любого натурального~$k$.
Сформулируйте критерий три-делимости многочлена.

\item
Пусть $f(x) = x^{p-1} + x^{p-2} + \ldots + x + 1$, где $p$~--- простое число.
Натуральное число~$m$ таково, что $p \divides m$.
Докажите, что любой простой делитель $f(m)$ взаимно прост с~$m (m - 1)$. 

\item
Докажите, что для любого простого~$p$ число $p^{p+1} + (p + 1)^p$ не~является
точным квадратом.
% Андрееску 7.1.2

\item
Найдите все тройки простых чисел $(p, q, r)$ такие, что
\[
    p \divides q^{r} + 1
\, , \quad
    q \divides r^{p} + 1
\, , \quad
    r \divides p^{q} + 1
\, . \]
% Андрееску 7.3.6

\item
Найдите все натуральные числа $m$ и~$n$, для которых $n^{m} = (n - 1)! + 1$.
% https://www.artofproblemsolving.com/community/c6t177f6h1261203

\item
Докажите, что не~существует таких натуральных чисел $p$ и~$n$, что $p$~---
простое, а~уравнение $x (x + 1) = p^{2n} y (y + 1)$ не~имеет решений
в~натуральных числах.
% https://www.artofproblemsolving.com/community/c6t177f6h1260538

\end{problems}

\endgroup % \def\divides


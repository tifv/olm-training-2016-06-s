% $date: 2016-06-15
% $timetable:
%   g11r2:
%     2016-06-15:
%       1:

\section*{Теорема Безу. Вокруг да около}

% $authors:
% - Владимир Викторович Трушков

\begin{problems}

\item
Найдите остатки от~деления многочлена $x^{81} + x^{27} + x^9 + x^3 + 1$ на%
\\
\subproblem $(x - 1)$;
\qquad
\subproblem $(x^2 - 1)$;
\qquad
\subproblem $(x^2 + 1)$;
\qquad
\subproblem $(x - 1)^2$.

\item
Многочлен $P(x)$ при делении на~$(x - 1)$ дает остаток~$2$, а~при делении
на~$(x - 2)$ дает остаток~$1$.
Какой остаток дает $P(x)$ при делении на~$(x - 1) (x - 2)$?

\item
Многочлен $P(x)$ при делении на~$(x^2 - 4)$ дает остаток $x + 1$,
а~на~$x^2 - 1$~--- остаток $x + 2$.
Найдите остаток при делении $P(x)$ на~$(x^2 - 4) (x^2 - 1)$.

\item
Докажите, что если значения двух многочленов, степени которых
не~превосходят $n$, совпадают в~$n + 1$ различных точках, то~эти многочлены
равны.

\item
Дан многочлен $P(x)$ такой, что многочлен $P(x^{n})$ делится на~$(x - 1)$.
Докажите, что многочлен $P(x)$ также делится на~$(x - 1)$.

\item
Известно, что многочлен $x^{n} + x + 1$ делится на~$x^2 + x + 1$.
Используя формулу Муавра, докажите, что $n$ есть число вида $3 k + 2$.

\item
Многочлен $P(x^3)$ делится на~многочлен $x^2 + x + 1$.
Используя комплексные числа, докажите, что многочлен $P(x)$ делится
на~многочлен $(x - 1)$.

\item
При каких $n$ многочлен $1 + x^2 + x^4 + \ldots + x^{2n-2}$ делится
на~$1 + x + x^2 + \ldots + x^{n-1}$?

\end{problems}


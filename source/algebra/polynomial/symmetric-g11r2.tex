% $date: 2016-06-16
% $timetable:
%   g11r2:
%     2016-06-16:
%       1:

\section*{Симметрические многочлены и системы}

% $authors:
% - Владимир Викторович Трушков

\begin{problems}

\item
Определите все значения параметра~$a$, при каждом из~которых три различных
корня уравнения $x^3 + (a^2 - 9 a) x^2 + 8 a x - 64 = 0$ образуют
геометрическую прогрессию.
Найдите эти корни.

\item
При каких значениях $a$ четыре корня уравнения
$x^4 + (a - 5) x^2 + (a + 2)^2 = 0$ являются последовательными членами
арифметической прогрессии?

\item
Пусть известно, что все корни некоторого уравнения $x^3 + p x^2 + q x + r = 0$
положительны.
Какому дополнительному условию должны удовлетворять его коэффициенты
$p$, $q$ и~$r$ для того, чтобы из~отрезков, длины которых равны этим корням,
можно было составить треугольник?

\item
\subproblem
Пусть $a$, $b$, $c$~--- стороны треугольника, $p$~--- его полупериметр,
а~$r$ и~$R$~--- радиусы вписанной и~описанной окружностей соответственно.
Составьте уравнение с~коэффициентами, зависящими от~$p$, $r$, $R$, корнями
которого являются числа $a$, $b$, $c$.
\\
\subproblem
Докажите равенство
\( \def\frac#1#2{#1/(#2)}
    \frac{1}{ab}+\frac{1}{ac}+\frac{1}{bc}=\frac{1}{2rR}
\).

\item
Длины сторон треугольника являются корнями кубического уравнения
с~рациональными коэффициентами.
Докажите, что длины высот треугольника являются корнями уравнения шестой
степени с~рациональными коэффициентами.

\item
Числа $x$, $y$, $z$ удовлетворяют системе
\begin{problemeq} \def\frac#1#2{#1/#2}
\left\{ \begin{aligned} &
    x + y + z = a
\, , \\ &
    \frac{1}{x} + \frac{1}{y} + \frac{1}{z} = \frac{1}{a}
\, . \\
\end{aligned} \right.
\end{problemeq}\\
Докажите, что хотя~бы одно из~этих чисел равно $a$.

\item
Пусть $a$, $b$, $c$~--- три различных числа.
Решите систему
\begin{problemeq}
\left\{ \begin{aligned} &
    z + a y + a^2 x + a^3 = 0
\, , \\ &
    z + b y + b^2 x + b^3 = 0
\, , \\ &
    z + c y + c^2 x + c^3 = 0
\, . \end{aligned} \right.
\end{problemeq}

\item
Выражение $x^{2009} + y^{2009}$ выразили через $\sigma_{1} = x + y$,
$\sigma_{2} = x y$, получили многочлен $g(\sigma_1, \sigma_2)$.
Найдите сумму коэффициентов этого многочлена.

\item
Назовем многочлен \emph{средиземноморским}, если он~имеет только действительные
корни и~имеет вид
\begin{align*}
    P(x)
={}&
    x^{10} - 20 x^9 + 135 x^8
    +\\&{}+
    a_7 x^7 + a_6 x^6 + a_5 x^5 + a_4 x^4 + a_3 x^3 + a_2 x^2 + a_1 x + a_0
\, .
\end{align*}
Коэффициенты $a_0$, \ldots, $a_7$~--- действительные числа.
Найдите наибольшее действительное число, которое может быть корнем
средиземноморского многочлена.

\end{problems}


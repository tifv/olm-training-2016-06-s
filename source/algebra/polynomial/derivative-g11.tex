% $date: 2016-06-19
% $timetable:
%   g11r2:
%     2016-06-21:
%       1:
%   g11r1:
%     2016-06-19:
%       1:

\section*{Производные}

% $authors:
% - Владимир Викторович Трушков

\begin{problems}

\item
Докажите, что уравнение $4 x^3 - 3 b x^2 + 2 c x - d = 0$ имеет 3
действительных корня тогда и~только тогда, когда существует действительные
числа $m$, $n$, $p$, $q$ такие, что
\[ \left\{ \begin{aligned} &
    b = m + n + p + q
, \\ &
    c = m n + m p + m q + n p + n q + p q
, \\ &
    d = m n p + m n q + m p q + n p q
.\end{aligned} \right. \]

\item
Многочлен четвертой степени $P(x)$ имеет четыре корня, попарные расстояния
между которыми не~меньше~$1$.
Докажите, что найдутся два корня $P'(x)$, находящиеся на~расстоянии
не~меньше~$1$.

\item
Найдите все действительные $a$ и~$b$ такие, что уравнение
$x^3 + a x^2 + b x + c = 0$ имеет не~более двух положительных корней при всех
значениях~$c$.

\item
Сколько существует многочленов вида $x^3 + a x^2 + b x + c$ таких, что
множество их~корней есть $\{ a, b, c \}$?

\item
Докажите, что при целых значениях $c$ уравнение
$x \cdot (x^2 - 1) \cdot (x^2 - 10) = c$ не~может иметь пяти целых корней.

\item
Докажите, что если все корни многочлена с~действительными коэффициентами
$P(x) = a_{n} x^{n} + \ldots + a_{0}$ действительны, то~и~все его производные имеют
лишь действительные корни.

\item
Пусть $P(x)$~--- многочлен степени~$n$
и~$P(a) \geq 0$, $P'(a) \geq 0$, \ldots,
$P^{(n-1)}(a) \geq 0$, $P^{(n)}(a) > 0$.
Докажите, что действительные корни уравнения $P(x) = 0$ не~превосходят $a$.

\item
Докажите, что многочлен
\begin{problemeq*}
    P(x)
=
    1 + x + x^2 / 2! + \ldots + x^n / n!
\end{problemeq*}
не~имеет кратных корней.

\item
Пусть
\begin{problemeq*}
    c_{0} + c_{1} / 2 + \ldots + c_{n} / (n + 1)
=
    0
\end{problemeq*}.
Докажите, что многочлен $c_{0} + c_{1} x + c_{2} x^{2} + \ldots + c_{n} x^{n}$
имеет хотя~бы один действительный корень.

\end{problems}


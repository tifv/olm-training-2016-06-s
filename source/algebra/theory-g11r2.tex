% $timetable:
%   g11r2: {}

\section*{Теоретический минимум}

\begingroup
    \def\iconst{\mathrm{i}}%
    \ifdefined\mathup
        \def\eulerphi{\mathup{\phi}}%
    \else
        \def\eulerphi{\upphi}%
    \fi
    \def\ord{\operatorname{ord}}%

\claim{Теорема Безу}
Остаток при делении многочлена $P(x)$ на~$(x - a)$ равен $P(a)$.

\claim{Теорема Эйзенштейна}
Рассмотрим многочлен с~целыми коэффициентами
\(
    P(x)
=
    a_{n} x^{n} + a_{n-1} x^{n-1} + \ldots + a_{2} x^{2} + a_{1} x + a_{0}
\).
Пусть $a_{n}$ не~делится на~простое~$p$;
пусть $a_{0}$, \ldots, $a_{n-1}$ делится на~$p$;
пусть $a_0$ не~делится на~$p^2$.
Тогда многочлен $P(x)$ неприводим, т.~е. его нельзя разложить в~произведение
двух многочленов ненулевой степени с~целыми коэффициентами.

\claim{Теорема Вильсона}
Число~$p$ простое тогда и~только тогда, когда $(p - 1)! + 1$ делится на~$p$.

\claim{Малая теорема Ферма}
Пусть $p$~--- простое, $(a, p) = 1$.
Тогда $a^{p-1} \equiv 1 \pmod p$.

\claim{Теорема Эйлера}
Пусть $(a, m) = 1$.
Тогда $a^{\eulerphi(m)} \equiv 1 \pmod m$.

Для простого~$p$ и~натурального~$n$ обозначим за $\ord_{p}(n)$ степень,
в~которой число~$p$ входит в~разложение числа~$n$ на~простые множители.

\claim{Лемма об~уточнении показателя}
Пусть $a$, $b$~--- различные целые числа, $k$~--- натуральное,
$p$~--- нечетное простое, являющееся делителем $(a - b)$ и~не~являющееся
делителем $a$.
Тогда
\[
    \ord_{p}(a^{k} - b^{k})
=
    \ord_{p}(a - b) + \ord_{p}(k)
\, . \]
Если $p = 2$, то, кроме уже описанных условий, должно быть выполнено условие,
что $(x - y)$ делится на~$4$.

\claim{Неравенство Коши}
Для всех неотрицательных чисел $a_{1}$, $a_{2}$, \ldots, $a_{n}$ имеет место
неравенство
\[
    \frac{a_{1} + a_{2} + \ldots + a_{n}}{n}
\geq
    \sqrt[n]{a_{1} \cdot a_{2} \cdot \ldots \cdot a_{n}}
\; . \]

\claim{Неравенство Коши--Буняковского}
Для действительных чисел $x_{1}$, $x_{2}$, \ldots, $x_{n}$,
$y_{1}$, $y_{2}$, \ldots, $y_{n}$ имеет место неравенство
\[
    (x_{1}^2 + x_{2}^2 + \ldots + x_{n}^2)
    \cdot
    (y_{1}^2 + y_{2}^2 + \ldots + y_{n}^2)
\geq
    (x_{1} y_{1} + x_{2} y_{2} + \ldots + x_{n} y_{n})^2
\; . \]

\claim{Неравенство Шура}
Пусть $x$, $y$, $z$~--- неотрицательные действительные числа.
Тогда для любого $r$ верно неравенство
\[
    x^{r} (x - y) (x - z) + y^{r} (y - x) (y - z) + z^{r} (z - x) (z - y)
\geq
    0
\, . \]

\claim{Транснеравенство}
Пусть $a_{1} \leq a_{2} \leq \ldots \leq a_{n}$
и~$b_{1} \leq b_{2} \leq \ldots \leq b_{n}$.
Тогда имеет место неравенство
\[
    a_{1} b_{n} + a_{2} b_{n-1} + \ldots + a_{n} b_{1}
\leq
    a_{1} c_{1} + a_{2} c_{2} + \ldots + a_{n} c_{n}
\leq
    a_{1} b_{1} + a_{2} b_{2} + \ldots + a_{n} b_{n}
\, , \]
если $с_{1}$, $c_{2}$, \ldots, $c_{n}$~--- произвольная перестановка
чисел $b_{1}$, $b_{2}$, \ldots, $b_{n}$.

\claim{Неравенство Йенсена}
Пусть $f(x)$~---непрерывно дифференцируемая выпуклая функция
на~отрезке $[a; b]$,
$x_{1}$, $x_{2}$, \ldots, $x_{n} \in [a; b]$.
Тогда
\[
    \frac{
        f(x_{1}) + f(x_{2}) + \ldots + f(x_{n})
    }{n}
\geq
    f\left(
        \frac{x_{1} + x_{2} + \ldots + x_{n}}{n}
    \right)
\; . \]

\claim{Формула Муавра}
Пусть
\(
    z
=
    r \cdot
    \bigl( \cos(\phi) + \iconst \sin(\phi) \bigr)
\);
$n$~--- натуральное.
Тогда
\[
    z^{n}
=
    r^{n} \cdot
    \bigl( \cos(n \phi) + \iconst \sin(n \phi) \bigr)
\; . \]

\endgroup % \def\iconst \def\eulerphi \def\ord


% $date: 2016-06-25
% $timetable:
%   g10r2:
%     2016-06-25:
%       2:

\section*{Разнобой-повторение}

% $authors:
% - Фёдор Львович Бахарев

\begin{problems}

\item
Точка~$X$ вне треугольника $ABC$ такова, что $A$ лежит внутри
треугольника $BXC$.
При этом $2 \angle XBA = \angle ACB$, $2 \angle XCA = \angle ABC$.
Докажите, что центры описанной и~вневписанной со~стороны~$BC$ окружностей
треугольника $ABC$ и~точка~$X$ лежат на~одной прямой.

\item
Найдите внутри треугольника точку с~минимальной суммой квадратов расстояний
до~вершин.

\item
На~окружности даны точки $A$, $B$, $C$, $D$ такие, что $AB$~--- диаметр круга,
а~$CD$~--- нет.
Докажите, что прямая, соединяющая точку пересечения касательных к~окружности
в~точках $C$ и~$D$ с~точкой пересечения прямых $AC$ и~$BD$, перпендикулярна
прямой~$AB$.

\item
Вписанная в~треугольник $ABC$ окружность касается сторон~$BC$, $CA$ и~$AB$
в~точках~$A_1$, $B_1$ и~$C_1$ соответственно.
Через точку~$A_1$ проведена прямая~$\ell$, перпендикулярная отрезку~$A A_1$.
Она пересекается с~прямой~$B_1 C_1$ в~точке~$X$.
Докажите, что прямая~$BC$ делит отрезок~$AX$ пополам.

\item
В~остроугольном треугольнике $ABC$ точки $I_{a}$ и~$I_{c}$~--- центры
вневписанных окружностей, $H$~--- основание высоты из~вершины~$B$.
Прямая~$I_{a} H$ пересекает $BC$ в~точке~$A'$, а~прямая~$I_{c} H$ пересекает
$AB$ в~точке~$C'$.
Докажите, что $A'C'$ проходит через центр вписанной окружности
треугольника $ABC$.

\item
Окружность касается сторон треугольника $AB$ и~$BC$ треугольника $ABC$
в~точках $D$ и~$E$, а~также внутренним образом описанной окружности
треугольника.
Докажите, что $DE$ проходит через центр вписанной окружности
треугольника $ABC$.

\item
В~сегмент, ограниченный хордой и~дугой $AB$ окружности, вписана окружность
$\omega$ с~центром~$I$.
Обозначим середину указанной дуги~$AB$ через $M$, а~середину дополнительной
дуги через $N$.
Из~точки~$N$ проведены две прямые, касающиеся $\omega$ в~точках $C$ и~$D$.
Противоположные стороны $AD$ и~$BC$ четырехугольника $ABCD$ пересекаются
в~точке~$Y$, а~его диагонали пересекаются в~точке~$X$.
Докажите, что точки $X$, $Y$, $I$ и~$M$ лежат на~одной прямой.

\item
Серединный перпендикуляр к~диагонали~$AC$ вписанного четырехугольника
$ABCD$ пересекает прямые $AD$ и~$CD$ в~точках $P$ и~$Q$.
Докажите, что биссектрисы углов $ABC$ и~$PBQ$ совпадают.

\item
Четырехугольник $ABCD$ вписан в~окружность с~центром~$O$ и~диаметром~$AC$.
Касательная в~точке~$C$ к~окружности пересекает прямую~$BD$ в~точке~$P$.
Луч~$AB$ пересекает отрезок~$PO$ в~точке~$E$.
Докажите, что $\angle DCE = 90^{\circ}$.

\end{problems}


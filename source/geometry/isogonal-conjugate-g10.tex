% $date: 2016-06-19
% $timetable:
%   g10r3:
%     2016-06-19:
%       2:
%   g10r2:
%     2016-06-20:
%       1:
%   g10r1:
%     2016-06-20:
%       2:

\section*{Изогональное сопряжение}

% $authors:
% - Фёдор Львович Бахарев

\definition
Дан треугольник $ABC$.
Две точки $P$ и~$Q$ называются \emph{изогонально сопряженными} относительно
треугольника $ABC$, если прямые $PA$ и~$QA$ симметричны относительно
биссектрисы угла $A$, прямые $PB$ и~$QB$ симметричны относительно биссектрисы
угла $B$, а~прямые $PC$ и~$QC$ симметричны относительно биссектрисы угла $C$.

\begin{problems}

\item
\subproblem
Докажите, что точка пересечения высот и~центр описанной окружности
треугольника изогонально сопряжены.
\\
\subproblem
Какие точки изогонально сопряжены самим себе?
\\
\subproblem
Касательные к~описанной окружности треугольника $ABC$ в~точках $B$ и~$C$
пересекаются в~точке~$P$.
Точка~$Q$ симметрична точке~$A$ относительно середины отрезка~$BC$.
Докажите, что $P$ и~$Q$ изогонально сопряжены относительно треугольника $ABC$.
\\
\subproblem
Докажите, что изогонально сопряжены точка, из~которой стороны треугольника
видны под углом $120^{\circ}$ \emph{(точка Торричелли)} и~точка, проекции
которой на~стороны треугольника образуют равносторонний треугольник
\emph{(изодинамический центр треугольника)}.
\\
\subproblem
Докажите, что изогонально сопряжены точка пересечения медиан треугольника
и~точка, для которой сумма квадратов расстояний до~его сторон минимальна.

\item
\subproblem
Пусть точки $P$ и~$Q$ изогонально сопряжены относительно треугольника $ABC$.
Пусть $x$, $y$ и~$z$~--- расстояния от~точки~$P$ до~прямых $BC$, $CA$ и~$AB$
соответственно, а~$x'$, $y'$ и~$z'$~--- расстояния от~точки~$Q$
до~прямых $BC$, $CA$ и~$AB$ соответственно.
Докажите, что $x \cdot x' = y \cdot y' = z \cdot z'$.
\\
\subproblem\jeolmlabel{geometry/isogonal-conjugate:sp:circles}%
Опустим из~точки~$P$ перпендикуляры на~стороны треугольника
(или их продолжения) и~рассмотрим окружность, проходящую через основания
перпендикуляров.
Докажите, что эта окружность совпадает с~окружностью, построенной таким~же
образом для точки $Q$.
\\
\subproblem
Выведите из~пункта~\jeolmsubref{geometry/isogonal-conjugate:sp:circles}, что
основания высот треугольника и~середины его сторон лежат на~одной окружности.

\item
\subproblem\jeolmlabel{geometry/isogonal-conjugate:sp:paskal-0}%
В~окружность вписан шестиугольник $ABCDEF$.
Отрезок~$AC$ пересекается с~отрезком~$BF$ в~точке~$X$, $BE$ с~$AD$~---
в~точке~$Y$ и~$CE$ и~$DF$ в~точке~$Z$.
Докажите, что треугольники $ABY$ и~$EDY$ подобны, причем точке
в~треугольнике $ABY$, изогонально сопряженной точке~$X$, соответствует
точка~$Z$ в~треугольнике $EDY$.
\\
\subproblem
Выведите из~пункта~\jeolmsubref{geometry/isogonal-conjugate:sp:paskal-0},
что точки $X$, $Y$ и~$Z$ лежат на~одной прямой.

\item
Докажите, что при изогональном сопряжении окружность, проходящая через
вершины $B$ и~$C$, отличная от~описанной, переходит в~окружность, проходящую
через $B$ и~$C$.

\item
В~трапеции $ABCD$ боковая сторона~$CD$ перпендикулярна основаниям,
$O$~--- точка пересечения диагоналей.
На~описанной окружности треугольника $OCD$ взята точка~$S$, диаметрально
противоположная точке~$O$.
Докажите, что $\angle BSC = \angle ASD$.

\item
Стороны треугольника $ABC$ видны из~точки~$T$ под углами $120^{\circ}$.
Докажите, что прямые, симметричные прямым $AT$, $BT$ и~$CT$ относительно прямых
$BC$, $CA$ и~$AB$ соответственно, пересекаются в~одной точке.

\end{problems}


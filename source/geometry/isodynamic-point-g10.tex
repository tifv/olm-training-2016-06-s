% $date: 2016-06-22
% $timetable:
%   g10r3:
%     2016-06-22:
%       2:
%   g10r2:
%     2016-06-24:
%       1:
%   g10r1:
%     2016-06-22:
%       1:

\section*{Изодинамические центры}

% $authors:
% - Фёдор Львович Бахарев

\begingroup
    \def\Ap{A\mspace{-1mu}p}%
%    \def\Ap{A_p}%

Точка~$\Ap$ называется \textbf{изодинамическим центром треугольника,} если ее
проекции на стороны являются вершинами равностороннего треугольника.

\begin{problems}

\item
\subproblem
Дан остроугольный треугольник $ABC$ и~его изодинамический центр $\Ap$, лежащий
внутри треугольника.
%Дан треугольник $ABC$ и~его изодинамический центр~$\Ap$.
Прямые $A \Ap$, $B \Ap$ и~$C \Ap$ повторно пересекают описанную окружность
треугольника $ABC$ в~точках $A_1$, $B_1$ и~$C_1$.
Докажите, что треугольник $A_1 B_1 C_1$ равносторонний.
\\
\subproblem
Сформулируйте и~докажите обратное утверждение.

\item
Даны числа $k_{1}$, $k_{2}$, \ldots, $k_{n}$, $c$ и~точки на~плоскости
$A_{1}$, $A_{2}$, \ldots, $A_{n}$.
Докажите, что множество точек~$X$, обладающих тем свойством, что
$k_{1} {A_{1} X}^2 + \ldots + k_{n} {A_{n} X}^2 = c$
\\
\subproblem
при $k_{1} + \ldots + k_{n} \neq 0$ является окружностью, точкой или пустым
множеством;
\\
\subproblem
при $k_{1} + \ldots + k_{n} = 0$ является прямой, плоскостью или пустым
множеством.

\end{problems}

\textbf{Окружность Аполлония} $\omega_{A}$ неравнобедренного треугольника
$ABC$~--- это геометрическое место точек~$M$, для которых $MB : MC = AB : AC$.
Аналогично определяются окружности Аполлония $\omega_B$ и~$\omega_C$.

\begin{problems}

\item
Докажите, что
\\
\subproblem
отрезок, соединяющий основания биссектрис внутреннего и~внешнего углов при
вершине~$A$, является диаметром окружности Аполлония $\omega_{A}$;
\\
\subproblem
центр окружности Аполлония $\omega_{A}$~--- точка пересечения касательной
к~описанной окружности в~вершине~$A$ с~прямой~$BC$;
\\
\subproblem
радикальная ось окружности~$\omega_{A}$ и~описанной окружности
треугольника $ABC$~--- это симедиана треугольника $ABC$, проведенная
из~вершины~$A$.

\item
\subproblem \jeolmlabel{geometry/isodynamic-point:pedal-triangle}%
Дана точка~$P$ и~треугольник $ABC$.
Докажите, что стороны педального треугольника, соответствующего точке~$P$,
вычисляются по~формулам
%$\frac{a \cdot PA}{2R}$, $\frac{b\cdot PB}{2R}$, $\frac{c\cdot PC}{2R}$.
$a \cdot PA / (2R)$, $b \cdot PB / (2R)$, $c \cdot PC / (2R)$.
В~задаче использованы стандартные обозначения для треугольника $ABC$.
\\
\subproblem
Выведите из~пункта~\jeolmsubref{geometry/isodynamic-point:pedal-triangle}
\emph{теорему Птолемея:}
\[
    PA \cdot BC + PB \cdot CA
\geq
    PC \cdot AB
\, , \]
причем равенство достигается только в~случае, когда точка~$P$ лежит
на~дуге~$AB$ описанной около треугольника $ABC$ окружности.
\\
\subproblem
Докажите, что изодинамический центр принадлежит всем трем окружностям
Аполлония.

\item
\subproblem
Докажите, что три окружности Аполлония имеют общую радикальную ось, проходящую
через центр описанной окружности.
\\
\subproblem
Докажите, что изодинамических центра у~неравенобедренного треугольника ровно
два.
\\
\subproblem
Докажите, что изодинамические центры симметричны (являются инверсными образами
друг друга) относительно описанной окружности треугольника $ABC$.

\item
Докажите, что изодинамические центры, центр описанной окружности
и~точка Лемуана лежат на~одной прямой \emph{(ось Брокара треугольника).}

\item
\subproblem
Пусть $\Ap$~--- изодинамический центр, лежащий внутри остроугольного
треугольника $ABC$.
Каков угол между описанной окружностью треугольника $ABC$ и~описанной
окружностью треугольника $\Ap B C$?
\\
\subproblem
Выразите углы $\angle A \Ap B$, $\angle B \Ap C$ и~$\angle C \Ap A$ через углы
треугольника $ABC$.

\end{problems}

\endgroup % \def\Ap


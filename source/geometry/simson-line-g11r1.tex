% $date: 2016-06-16
% $timetable:
%   g11r1:
%     2016-06-16:
%       1:

\section*{Прямая Симсона}

% $authors:
% - Алексей Вадимович Доледенок

\claim{Прямая Симсона}
Рассмотрим треугольник $ABC$ и~произвольную точку~$P$.
Основания перпендикуляров из~точки~$P$ на~прямые $AB$, $AC$, $BC$ лежат
на~одной прямой тогда и~только тогда, когда точка~$P$ лежит на~описанной
окружности треугольника $ABC$.

\begin{problems}

%\item
%Даны точки $A$, $B$, $C$, лежащие на~одной прямой, и~точка~$P$ вне этой прямой.
%Докажите, что центры описанных окружностей треугольников $ABP$, $ACP$, $BCP$
%и~точка~$P$ лежат на~одной окружности.

\item
Пусть $A A_1$, $B B_1$, $C C_1$~--- высоты остроугольного треугольника $ABC$.
Докажите, что основания перпендикуляров из~точки~$A_1$ на~прямые
$AB$, $AC$, $B B_1$, $C C_1$ лежат на~одной прямой.

\item
Две окружности пересекаются в~точках $A$ и~$B$.
Через точку~$A$ проходит прямая, пересекающая окружности в~точках $C$ и~$D$.
Точки $P$ и~$Q$~--- проекции точки~$B$ на~касательные к~окружностям,
проведенным в~точках $C$ и~$D$.
Докажите, что прямая~$PQ$ касается окружности, построенной на~$AB$ как
на~диаметре.

\item
В~треугольнике $ABC$ угол~$A$ равен $60^{\circ}$.
Пусть $B B_1$ и~$C C_1$~--- биссектрисы.
Докажите, что точка, симметричная $A$ относительно $B_1 C_1$, лежит
на~прямой~$BC$.
%\emph{Указание. $A B_1 I C_1$ вписанный.
%Рассмотрим прямую Симсона точки $A$.}

\item
В~треугольнике $ABC$ провели биссектрису $A A_1$, из~точки~$A_1$ опустили
перпендикуляры $A_1 X$ и~$A_1 Y$ на~стороны $AB$ и~$AC$ соответственно.
На~отрезке~$XY$ выбрана точка~$M$ так, что $M A_1 \perp BC$.
Докажите, что точка~$M$ лежит на~медиане треугольника $ABC$, проведенной
из~вершины~$A$.

%\item
%Дан треугольник $ABC$.
%Рассматриваются прямые~$l$, обладающие следующим свойством: три прямые,
%симметричные $l$ относительно сторон треугольника, пересекаются в~одной точке.
%Докажите, что все такие прямые проходят через одну точку.

%\item \claim{Прямая Обера}
%Четыре прямых общего положения образуют в~пересечении четыре треугольника.
%Докажите, что ортоцентры этих треугольников лежат на~одной прямой.

\item
В~треугольнике $ABC$ из~произвольной точки~$P$ дуги~$BC$ описанной окружности
треугольника $ABC$ опущены перпендикуляры $PX$ и~$PY$ на~стороны $AB$ и~$BC$.
Пусть $M$ и~$N$~--- середины отрезков $AC$ и~$XY$.
Докажите, что $\angle MNP = 90^{\circ}$.
%\emph{Указание.
%Пусть $Z$~--- проекция $P$ на~$AC$.
%Тогда из~подобия треугольников $PXY$ и~$PAC$ следует, что
%$\angle PMC = \angle PNY = \angle PNZ$.
%Отсюда $MNPZ$ вписанный.}

\item
Пусть $H$~--- ортоцентр треугольника $ABC$, $P$~--- произвольная точка
на~описанной окружности $ABC$.
Докажите, что прямая Симсона точки~$P$ делит отрезок~$PH$ пополам.

\item
\subproblem
Точка~$P$ движется по~описанной окружности треугольника $ABC$.
Докажите, что при этом прямая Симсона точки~$P$ относительно треугольника $ABC$
поворачивается на~угол, равный половине угловой величины дуги, пройденной
точкой~$P$.

\subproblem
Докажите, что прямые Симсона двух диаметрально противоположных точек описанной
окружности треугольника $ABC$ перпендикулярны, а~их точка пересечения лежит
на~окружности Эйлера.

\item
\subproblem
Докажите, что на~описанной окружности треугольника существует ровно три точки
таких, что их прямая Симсона касается окружности Эйлера, причем они образуют
равносторонний треугольник.
\\
\subproblem \emph{Теорема Морлея.}
В~треугольнике $ABC$ провели трисектрисы углов.
Пусть $A_1$~--- точка пересечения ближайших к~стороне~$BC$ трисектрис
углов $B$ и~$C$.
Аналогично определяются точки $B_1$ и~$C_1$.
Докажите, что треугольник $A_1 B_1 C_1$ правильный,
а~$\angle A B_1 C_1 = 60^{\circ} + \angle C / 3$.
\\
\subproblem
Докажите, что на~описанной окружности треугольника $ABC$ найдется три точки
таких, что их прямые Симсона касаются окружности Эйлера, причем эти точки
образуют правильный треугольник, стороны которого параллельны сторонам
треугольника Морлея.

% Шарыгинская олимпиада-2014, финал.
\item
Дан фиксированный треугольник $ABC$.
Пусть $D$~--- произвольная точка в~плоскости треугольника, не~совпадающая с~его
вершинами.
Окружность с~центром в~$D$, проходящая через $A$, пересекает вторично прямые
$AB$ и~$AC$ в~точках $A_{b}$ и~$A_{c}$ соответственно.
Аналогично определяются точки $B_{a}$, $B_{c}$, $C_{a}$ и~$C_{b}$.
Точку~$D$ назовем хорошей, если точки
$A_{b}$, $A_{c}$, $B_{a}$, $B_{c}$, $C_{a}$ и~$C_{b}$
лежат на~одной окружности.
Сколько может оказаться точек, хороших для данного треугольника $ABC$?

\end{problems}


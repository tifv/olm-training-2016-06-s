% $date: 2016-06-25

\section*{Математический анализ}

% $authors:
% - Алексей Яковлевич Канель-Белов

% $style:
% - /
% - verbatim: \renewcommand\jeolmgroupnames{{\large Анонс лекции}}
%   provide: renewcommand-jeolmgroupnames

\leavevmode\hfill\begin{minipage}{0.45\linewidth} \em \small
На~экзамене Л.\,Камынин сделал замечание студенту, что тот отвечает не~по~его
лекции.
\\--- Что я сдаю? Матанализ или Ваши лекции?
\\--- Вы должны сдавать мои лекции.
\\--- А~где тут сдают математический анализ?
\end{minipage}

Что такое матанализ?

Это не~ерунда вроде вычисления пределов типа такого
$\lim\limits_{x \to 1} \frac{x^2 - 1}{x - 1}$
по~определению, и~не~интеллектуальный мусор типа дедекиндовых сечений,
заполняющий университетские и~матшкольные программы.

Аналитическое мышление~--- это видение главного и~второстепенного, того, чем
можно пренебречь.
И~в~этом (а~не~в~вычислениях) матанализ смыкается с~физикой.

Предполагается рассказать несколько сюжетов, относящихся к~физике, вероятности
и~собственно теории фунций действительного переменного на~наглядном
уровне.

\begingroup\bfseries\large
Сегодня с~19:30 до~20:30 в~каб.~9 состоится лекция
доктора физ.-мат. наук Алексея Яковлевича Канеля-Белова.
\endgroup


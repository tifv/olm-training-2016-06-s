% $date: 19--22 июня 2016 г.

\section*{Решение головоломок при помощи линейного программирования}

% $authors:
% - Алексей Сергеевич Тарасов

% $style[-announcement]:
% - .[announcement]

\begingroup
    \providecommand\url{\texttt}

В~спецкурсе будет рассказано о~задачах линейного программирования (непрерывной
и~дискретной), алгоритме симплекс метода.
Будут приведены различные примеры решения головоломок (например, судоку)
и~практических задач при помощи сведения к~задачам линейного программирования.
Если останется время, будет рассказано о~более общих задачах оптимизации~---
нелинейного программирования, смешанно-целочисленного программирования.

На~спецкурсе~бы будем в~числе прочего программировать на~языке Python, поэтому
вы должны знать хоть немного либо этот язык, либо любой другой, но~достаточно
хорошо.

\textbf{Техническое требование.}
При приходе на~спецкурс нужно иметь не~менее половины ноутбука (то~есть иметь
или ноут, или друга с~ноутом), на~который нужно поставить \textsf{glpk}, скачав
отсюда \url{http://winglpk.sourceforge.net}.
Если возникают технические вопросы по~ПО~--- обратитесь
к~Тихонову Юлию Васильевичу.

\endgroup % \def\url


% $date: 19--22 июня 2016 г.

\section*{Три истории про мозаики}

% $authors:
% - Иван Викторович Митрофанов

% $style[-announcement]:
% - .[announcement]

Сначала мы поговорим о~разрезаниях фигур на~обычные доминошки $2 \times 1$
и~обнаружим Хитрый Инвариант, позволяющий быстро понимать, можно~ли осуществить
разрезание.

Потом мы рассмотрим замощения на~гексагональной решетке и, используя идеи
теории групп, получим Ещё Более Хитрый Инвариант, позволяющий доказать,
например, такой факт: если $n (n + 1) / 2$ точек образуют правильный
треугольник, то~их нельзя разбить на~тройки идущих в~ряд.

И, наконец, поймем, что даже Самые Хитрые Инварианты и~мощнейшие компьютеры
не~помогут решить некоторые задачи о~разрезании бесконечной плоскости, потому
что на~них нельзя найти ответ принципиально.


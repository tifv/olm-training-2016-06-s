% $date: 19--22 июня 2016 г.

\section*{Комплексные числа в геометрии}

% $authors:
% - Владимир Натанович Дубровский

% $style[-announcement]:
% - .[announcement]

Есть три геометрических интерпретации комплексных чисел: как точек, как
векторов и~как преобразований.
В~сочетании с~алгеброй они образуют очень удобный аппарат для решения
геометрических задач.
Его описанию и~применениям и~будет посвящен этот курс.

Большая часть курса будет посвящена связи между комплексными числами
и~преобразованиями: мы опишем всевозможные виды преобразований подобия
и~научимся
находить их композиции, докажем «основную теорему о~собственно подобных
фигурах», теорему о~косом пропеллере, теорему Наполеона и~ее многочисленные
обобщения – теоремы об~ушастых многоугольниках.

Будут рассмотрены и~другие примеры представления геометрических понятий
и~свойств с~помощью комплексных чисел и~их применении к~решению задач, многие
из~которых предлагались на~разных олимпиадах.

Рассказ будет сопровождаться красивыми динамическими картинками.

Знать, что такое комплексные числа, не~обязательно.


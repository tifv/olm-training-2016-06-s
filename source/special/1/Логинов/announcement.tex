% $date: 15--18 июня 2016 г.

\section*{Многочлены деления круга и теорема Дирихле}

% $authors:
% - Константин Валерьевич Логинов

% $style[-announcement]:
% - .[announcement]

Многие знают, что многочлен $(x^n - 1) / (x - 1)$ неприводим, когда $n$
является простым числом.
А~что для других $n$?
Оказывается, в~общем случае он раскладывается в~произведение некоторого числа
так называемых циклотомических многочленов, или многочленов деления круга.
Они обладают многими интересными свойствами, а~кроме того, глубоко связаны
с~теорией чисел.
Познакомившись с~ними, мы изучим (или вспомним), сколько бывает корней
из~единицы над полем комплексных чисел, что такое критерий Эйзенштейна, функция
Мебиуса, а~также докажем частный случай знаменитой теоремы Дирихле, гласящей,
что в~любой арифметической прогрессии можно найти бесконечно много простых
чисел.

Приветствуется (но~не~является обязательным) знакомство с~комплексными
числами, функцией Эйлера, а~также понятием группы.


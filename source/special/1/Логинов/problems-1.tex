% $date: 15--18 июня 2016 г.

\section*{Круговые многочлены, часть 1}

% $authors:
% - Константин Валерьевич Логинов

\begingroup
    \ifdefined\mathup
        \def\piconst{\mathup{\pi}}%
    \else
        \def\piconst{\uppi}%
    \fi

\begin{problems}

\itemy{0}
Выразите $974$ в~виде суммы двух квадратов, зная, что $13 = 2 ^ 2 + 3 ^ 2$,
$74 = 5^2 + 7^2$.

\item
Найти все значения $\sqrt[3]{-i}$.

\item
Докажите тождества
\begin{align*}
    C_{n}^{1} - C_{n}^{3} + C_{n}^{5} - \ldots
& =
    2^{n/2} \cos \frac{n \piconst}{4}
, \\
    C_{n}^{0} - C_{n}^{2} + C_{n}^{4} - \ldots
& =
    2^{n/2} \sin \frac{n \piconst}{4}
. \end{align*}

\item
Докажите, что многочлен
\(
    (\cos\theta + x \sin\theta)^{n}
    - \cos(n \theta) - x \sin(n \theta)
\) делится на~$x^2 + 1$.

\item
Докажите, что многочлен
\(
    x^{4a} + x^{4b + 1} + x^{4c + 2} + x^{4d + 3}
\), где $a, b, c, d$~--- натуральные числа, делится на~$x^3 + x^2 + x + 1$.

\item
Найдите наибольший общий делитель многочленов $(x^{n} - 1)$ и~$(x^{m} - 1)$.

\item
Докажите, что
\[
    \sin \frac{\piconst}{n} \cdot \sin \frac{2 \piconst}{n}
    \cdot \ldots \cdot
    \sin \frac{(n - 1) \piconst}{n}
=
    \frac{n}{2^{n-1}}
\, . \]

\end{problems}


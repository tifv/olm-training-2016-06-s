% $date: 15--18 июня 2016 г.

\section*{Классические и новые задачи теории Рамсея}

% $authors:
% - Андрей Михайлович Райгородский

% $style[-announcement]:
% - .[announcement]

Многие (если не~все) знают, что среди любых шести человек либо есть трое
попарно знакомых, либо есть трое попарно незнакомых.
Утверждения такого типа относятся к~теории Рамсея, и~это тоже многие знают.
Разумеется, в~нашем курсе мы напомним эти факты.
Однако сделаем мы в~итоге гораздо больше (а~то~какой~же иначе катарсис?).
В~частности, мы изучим такой замечательный и~современный объект, как
<<двудольные числа Рамсея>>, а~если повезет, то~затронем и~совсем свежие
<<дистанционные числа Рамсея>>.


% $date: 15--18 июня 2016 г.

\section*{Вокруг дерева Фарея}

% $authors:
% - Николай Германович Мощевитин

% $style[-announcement]:
% - .[announcement]

Все вы должны хорошо знать, как надо складывать рациональные дроби, то~есть
правило
\[
    \frac{a}{b} + \frac{c}{d}
=
    \frac{a d + b c}{b d}
\; . \]
Оказывается, иногда уместно <<складывать>> дроби по-другому, вот так:
\[
    \frac{a}{b} \oplus \frac{c}{d}
=
    \frac{a + c}{b + d}
\; . \]
Эта процедура, называющаяся \emph{взятием медианты,} в~частности, позволяет
построить все рациональные числа из~0 и~1 в~качестве вершин бинарного
дерева~--- дерева Фарея.
Дерево Фарея порождает два естественных объекта~--- ряды Фарея
и~последовательности Штерна--Броко.
Вопросы о~распределении элементов рядов Фарея связаны со~знаменитой гипотезой
Римана о~нулях дзета-функции.
Последовательности Штерна--Броко связаны с~$?$-функцией Минковского.


% $date: 24--27 июня 2016 г.

\section*{Комплексные числа, кватернионы, \ldots}

% $authors:
% - Фёдор Юрьевич Попеленский

% $style[-announcement]:
% - .[announcement]

\begingroup
    \def\iconst{\mathrm{i}}%
    \def\jconst{\mathrm{j}}%
    \def\kconst{\mathrm{k}}%

Многие из~вас уже познакомились с~комплексными числами и~кое-какими важными
математическими результатами, которые с~ними связаны
(если вы продолжите заниматься математикой, то~таких результатов,
без преувеличения, хватит еще на~несколько лет).

Комплексные числа получаются присоединением к~вещественным числам одной мнимой
единицы~$\iconst$.
Естественный вопрос, какие еще бывают <<гиперкомплексные числа>>, оказался
не~так-то прост.
Долгое время безуспешно искали <<гиперкомплексные числа>>, которые получаются
присоединением двух мнимых единиц.
Но~в~конце концов Гамильтону удалось придумать \emph{кватернионы}~--- это
<<гиперкомплексные числа>>, которые получаются добавлением к~обычным
вещественным числам трех мнимых единиц $\iconst$, $\jconst$, $\kconst$.

Мы подробно обсудим кватернионы и~выясним, как с~их помощью можно представлять
вращения трехмерного пространства.
Также мы разберемся, почему не~удалось придумать гиперкомплексные числа,
которые получаются присоединением всего двух мнимых единиц.
И~с~самыми стойкими выясним, бывают~ли гиперкомплексные числа, которые
получаются присоединением к~вещественным числам другого количества мнимых
единиц.
Наконец, если останется время, познакомимся в~алгебрами Клиффорда и~октавами.

Знакомство с~комплексными числами будет полезно, но~не~обязательно.
Все необходимые понятия будут введены по~ходу рассказа.

\endgroup % \def\iconst \def\jconst \def\kconst


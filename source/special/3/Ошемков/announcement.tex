% $date: 24--27 июня 2016 г.

\section*{Что такое проективная геометрия}

% $authors:
% - Андрей Александрович Ошемков

% $style[-announcement]:
% - .[announcement]

Название <<проективная геометрия>> связано с~отображением проектирования
из~точки.
Помимо математиков свойства такого отображения изучали художники, архитекторы,
инженеры, поскольку это связано с~перспективой, т.~е. с~более правильным
изображением того, что мы наблюдаем.
В математике это приводит к~тому, что к~<<обычным>> точкам (например, на~прямой
или на~плоскости) надо добавить бесконечно удаленные.
Оказывается, что после этого формулировки и~доказательства многих
геометрических фактов не~только упрощаются, но~и~становятся более
естественными, понятными и~не~требуют рассмотрения различных случаев, оговорок
и~исключений.

В~курсе предполагается обсудить, что такое проективная прямая и~проективная
плоскость, как устроены их проективные преобразования.
Оказывается, что связанные с~этим понятия (двойное отношение, поляра,
проективная двойственность) позволяют решать многие геометрические задачи
чрезвычайно просто и~красиво (а~также легко <<изобретать>> новые задачи
и~теоремы).
Мы обсудим как классические теоремы Паппа, Дезарга, Паскаля, Брианшона,
Понселе, так и~множество других задач, которые можно решить подобными методами.

Никаких специальных знаний кроме школьного курса геометрии не~предполагается,
но~хорошее знание того, как устроены на~плоскости движения и~гомотетии, было~бы
полезно.


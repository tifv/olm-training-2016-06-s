% $date: 24--27 июня 2016 г.

\section*{Ряды}

% $authors:
% - Евгений Витальевич Щепин

% $style[-announcement]:
% - .[announcement]

Одной из~важнейших идей математического анализа является представление функции
в~виде бесконечной суммы мономов, грубо говоря, в~виде бесконечного многочлена.
На~спецкурсе мы увидим, что такое представление имеют почти все <<известные>>
функции: рациональные функции, показательные и~тригонометрические.
С~помощью этих разложений мы изучим различные способы вычисления бесконечных
(и~некоторых конечных) сумм.
Если останется время, поговорим также о~бесконечных произведениях
и~Гамма-функции, являющейся обобщением понятия факториала.

Для слушателей желательно, но~не~обязательно знакомство с~комплексными числами
и~производной.


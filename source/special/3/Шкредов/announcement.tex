% $date: 24--27 июня 2016 г.

\section*{Введение в теорию сумм произведений}

% $authors:
% - Илья Дмитриевич Шкредов

% $style[-announcement]:
% - .[announcement]

Пусть $A$~--- конечное подмножество целых чисел.
Тогда по~$A$ можно сформировать два других множества:
\emph{сумму} $A + A = \{ a_{1} + a_{2} \colon a_{1}, a_{2} \in A \}$
и~\emph{произведение} $A \cdot A = \{ a_{1} \cdot a_{2} \colon a_{1}, a_{2} \in A \}$.
%\begin{gather*}
%\text{\emph{сумму}} \quad
%    A + A = \{ a_{1} + a_{2} \colon a_{1}, a_{2} \in A \}
%\\
%\text{и~\emph{произведение}} \quad
%    A \cdot A = \{ a_{1} \cdot a_{2} \colon a_{1}, a_{2} \in A \}
%\, . \end{gather*}
%\begin{align*}
%\text{\emph{сумму}} \quad
%    A + A &= \{ a_{1} + a_{2} \colon a_{1}, a_{2} \in A \}
%\\
%\text{и~\emph{произведение}} \quad
%    A \cdot A &= \{ a_{1} \cdot a_{2} \colon a_{1}, a_{2} \in A \}
%\, . \end{align*}
Феномен сумм произведений был открыт Эрдёшем и~Семереди в~80-х годах
20-го века.
Он состоит в~том, что никакое множество не~может иметь одновременно малые
сумму и~произведение.
В~21-м веке было осознано, что данное утверждение сильно связано с~геометрией
и~может быть перенесено на~другие числовые системы.
Наш спецкурс представляет собой элементарное введение в~данный предмет.

Никаких знаний, выходящих за~рамки школьной программы не~требуется.
Приглашаются все желающие!


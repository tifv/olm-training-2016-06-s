% $date: 24--27 июня 2016 г.

\section*{Коники. Задачи}

% $authors:
% - Фёдор Львович Бахарев

\begin{problems}

\item
\subproblem
Какими неравенствами задаются области, на~которые гипербола разбивает
плоскость, и~почему?
\\
\subproblem
Какими неравенствами задаются области, на~которые парабола разбивает плоскость,
и~почему?

\item
Найдите геометрическое место центров окружностей, касающихся двух данных.

\item
Докажите, что уравенение
\begin{problemeq*} \displaystyle
    \frac{x^2}{a^2} + \frac{y^2}{b^2} = 1
\end{problemeq*},
где $a > b > 0$,
задает эллипс с~фокусами $(\pm\sqrt{a^2 - b^2}, 0)$.

\item
Докажите, что уравнение $x y = 1$ задает гиперболу и~найдите ее фокусы.

\item
Докажите фокальное свойство параболы.

\item
Докажите, что семейства софокусных гиперболи и~эллипсов ортогональны.

\item
Докажите изогональное свойство гиперболы.

\item
Постройте с~помощью циркуля и~линейки вершину параболы.

\item
Пусть точка~$X$ движется по~параболе, перпендикуляр к~касательной в~точке~$X$
пересекает ее ось в~точке~$Y$, а~$Z$~--- проекция точки~$X$ на~ось.
Докажите, что длина отрезка~$ZY$ не~меняется.

\item
Парабола вписана в~угол $PAQ$.
Найдите геометрическое место середин отрезков, высекаемых сторонами угла
на~касательных к~параболе.

\item
Докажите, что прямые Симсона двух диаметрально противоположных точек
на~описанной окружности треугольника $ABC$ перпендикулярны, а~их точка
пересечения лежит на~окружности девяти точек треугольника.

\item
Докажите, что прямая Симсона точки~$P$ лежащей на~окружности, описанной около
треугольника $ABC$, перпендикулярна прямым, симметричным $PA$, $PB$ и~$PC$
относительно сторон треугольника $BC$, $CA$ и~$AB$ соответственно.

\item
На~равносторонней гиперболе взята произвольная точка~$P$.
Обозначим через~$Q$ точку, симметричную точке~$P$ относительно центра этой
гиперболы.
Окружность с~центром~$P$ и~радиусом~$PQ$ пересекает гиперболу еще в~трех точках
$A$, $B$ и~$C$.
Докажите, что треугольник $ABC$ правильный.

\item
Пусть $P$~--- это центр равносторонней гиперболы, проходящей через вершины
вписанного четырехугольника $ABCD$.
Докажите, что $P$ лежит на~прямой, соединяющей центр описанной окружности
и~центр тяжести четырехугольника $ABCD$.

\item
Парабола касается сторон треугольника $ABC$ в~точках $A'$, $B'$ и~$C'$.
Докажите, что точка пересечения прямой, проходящей через точку~$C'$
и~параллельной оси параболы, с~прямой $A'B'$ лежит на~медиане $C M_{c}$.

\end{problems}


% $date: 24--27 июня 2016 г.

\section*{Коники}

% $authors:
% - Фёдор Львович Бахарев

% $style[-announcement]:
% - .[announcement]

Кривые второго порядка, или коники, традиционно считаются объектом
аналитической геометрии, при этом из~их геометрических свойств упоминаются,
в~лучшем случае, только оптические.
Между тем, эти кривые обладают рядом других весьма красивых свойств, большая
часть которых может быть доказана методами элементарной геометрии.
Кроме того, коники могут применяться для решения геометрических задач,
на~первый взгляд никак с~ними не~связанных.

В~спецкурсе предполагается рассказать наиболее интересные факты, связанные
с~кривыми второго порядка.
Для начала мы разберемся с~элементарными свойствам коник, большая часть которых
широко известна.
Некоторые несложные, но~важные утверждения будут предложены в~качестве задач.
Для понимания дальнейшего изложения хорошо~бы разбираться хотя~бы на~начальном
уровне в~следующих вещах класической геометрии:
изогональное сопряжение, поляры, теорема Паскаля, инверсия, симедианы,
гармонический четырехугольник.
Если вы слышали об~аффинных и~проективных преобразованиях, то~это вам тоже
пойдет на~руку.

Зачет будет выставляться по~решенным задачам из~списка, выданного во~время
спецкурса.


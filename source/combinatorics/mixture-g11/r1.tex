% $date: 2016-06-26
% $timetable:
%   g11r1:
%     2016-06-26:
%       1:

\section*{Разнобой по комбинаторике}

% $authors:
% - Владимир Алексеевич Брагин
% - Иван Викторович Митрофанов

\begingroup
    \def\abs#1{\lvert #1 \rvert}%

\begin{problems}

\item
В~марсианском алфавите $k$ букв.
Два слова называются \emph{похожими,} если в~них поровну букв и~они отличаются
ровно одной буквой (в~одном разряде, например: ТРУКС и~ТРИКС).
Докажите, что все слова можно разбить на~$k$~групп так, чтобы в~одной группе
не~было похожих слов.

\item
На~плоскости отмечено несколько точек, не~лежащих на~одной прямой, и~около
каждой написано число.
Для любой прямой, проходящей через две или более отмеченных точек, сумма всех
чисел, написанных около этих точек, равна 0.
Докажите, что все числа равны 0.

\item
Сколько существует перестановок из~$n$ элементов, не~имеющих неподвижных
элементов?

\item
Петя и~Вася играют в~игру, правила которой таковы.
Петя загадывает натуральное число ~$x$ c суммой цифр $2012$.
За~один ход Вася выирает любое натуральное число~$a$ и~узнает сумму цифр
числа $\abs{x - a}$.
Какое наименьшее число ходов потребуется Васе, чтобы гарантрованно найти $x$?

\item
В~каждой клетке бесконечного клетчатого листа расставлены действительные числа.
Оказалось, что сумма чисел в~каждом квадрате по~модулю не~превосходит 1.
Докажите, что сумма чисел в~любом прямоугольнике не~превосходит 1000.

\end{problems}

\endgroup % \def\abs


% $date: 2016-06-27
% $timetable:
%   g11r2:
%     2016-06-27:
%       1:

\section*{Разнобой по комбинаторике}

% $authors:
% - Владимир Алексеевич Брагин
% - Иван Викторович Митрофанов

\begin{problems}

\item
У~Ивана есть 3~куска хлеба, 4 сыра и~7 колбасы.
Сколькими способами он может сделать бутерброд так, чтобы
\\
\subproblem никакие два куска колбасы не~лежали рядом;
\\
\subproblem никакие два одинаковых продукта не~находились рядом?

\item
В~клетках шахматной доски расставлены плюс и~минус единицы.
Оказалось, что любые два соседних столбца либо равны, либо противоположны.
Докажите, что для строчек выполняется аналогичное условие.

\item
В~марсианском алфавите $k$ букв.
Два слова называются \emph{похожими,} если в~них поровну букв и~они отличаются
ровно одной буквой (в~одном разряде, например: ТРУКС и~ТРИКС).
Докажите, что все слова можно разбить на~$k$~групп так, чтобы в~одной группе
не~было похожих слов.

\item
На~плоскости отмечено несколько точек, не~лежащих на~одной прямой, и~около
каждой написано число.
Для любой прямой, проходящей через две или более отмеченных точек, сумма всех
чисел, написанных около этих точек, равна 0.
Докажите, что все числа равны 0.

\item
Последовательность натуральных чисел $\{ a_{n} \}$ обладает тем свойством, что
для любого $k$ в~ней содержится ровно $k$~делителей числа~$k$.
Докажите, что в~ней содержится любое натуральное число хотя~бы один раз.

\end{problems}


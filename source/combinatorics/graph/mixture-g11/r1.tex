% $date: 2016-06-25
% $timetable:
%   g11r1:
%     2016-06-25:
%       1:

\section*{Разнобой}

% $authors:
% - Владимир Алексеевич Брагин
% - Иван Викторович Митрофанов

\begin{problems}

\item
В~графе 16~вершин, степень каждой вершины равна 6.
У~любых двух вершин ровно два общих соседа.
Сколько в~этом графе циклов длины 3?

\item
В~классе 21~ученик.
У~всех, кроме Кости, разное число друзей в~классе.
Сколько друзей могло быть у~Кости?

\item
В~связном графе ребра покрашены в~два цвета так, что из~каждой вершины выходит
поровну рёбер двух цветов.
Докажите, что из~любой вершины в~любую существует путь, цвета ребер которого
чередуются.

\item
В~деревне Сплетня 100 жителей.
У~каждого жителя не~менее трех друзей среди остальных.
 В~первый день один из~жителей узнал интереснейшую новость и~тут~же поделился
ей со~своими друзьями.
Каждый узнающий на~следующий день тоже делился новостью с~остальными своими
друзьями.
Известно, что когда-то новость узнали все.
А~через какое наибольшее количество дней это впервые могло случиться?

\item
Вершины выпуклого многогранника можно обойти одним циклическим маршрутом.
Докажите, что грани можно правильным образом покрасить в~4 цвета.

\item
Степень каждой вершины графа не~превосходит $n$, при этом среди любых
$m$~вершин есть две соединенные ребром.
При каком наибольшем числе вершин такое возможно?

\item
В~стране 120 городов.
Некоторые пары городов соединены дорогами, не~проходящими через другие города.
Из~каждого города выходит хотя~бы три дороги.
Докажите что существует несамопересекающийся циклический маршрут состоящий
не~более чем из~11 городов.

\item
В~графе $E$ ребер и~$T$ треугольников.
Докажите, что $9 T^2 \leq 2 E^3$.

\end{problems}


% $date: 2016-06-25
% $timetable:
%   g11r2:
%     2016-06-25:
%       2:

\section*{Разнобой}

% $authors:
% - Владимир Алексеевич Брагин
% - Иван Викторович Митрофанов

\begin{problems}

\item
Волейбольная сетка состоит из~50 вертикальных и~300 горизонтальных клеток.
Какое наибольшее количество веревочек между узлами можно перерезать так, чтобы
сетка не~развалилась?

\item
В~графе 16~вершин, степень каждой вершины равна 6.
У~любых двух вершин ровно два общих соседа.
Сколько в~этом графе циклов длины 3?

\item
Чтобы отвести на~завтрак, 100 детей построили парами.
На~обратном пути из~столовой их снова построили парами, возможно, составленными
по-другому.
При каком наибольшем $n$ наверняка можно выбрать $n$~детей, никакие два из~них
не~были в~одной паре?

\item
Существует~ли такой граф, у~которого ровно два остовных дерева?

\item
В~классе 21~ученик.
У~всех, кроме Кости, разное число друзей в~классе.
Сколько друзей могло быть у~Кости?

\item
В~связном графе ребра покрашены в~два цвета так, что из~каждой вершины выходит
поровну рёбер двух цветов.
Докажите, что из~любой вершины в~любую существует путь, цвета ребер которого
чередуются.

\item
На~турнир приехали 100 человек.
Известно, что среди любых 50 из~них есть человек, знакомый с~остальными 49.
Для какого наибольшего $k$ можно утверждать, что в~этой компании найдутся
$k$~человек, знакомых друг с~другом?

\item
В~деревне Сплетня 100 жителей.
У~каждого жителя не~менее трех друзей среди остальных.
 В~первый день один из~жителей узнал интереснейшую новость и~тут~же поделился
ей со~своими друзьями.
Каждый узнающий на~следующий день тоже делился новостью с~остальными своими
друзьями.
Известно, что когда-то новость узнали все.
А~через какое наибольшее количество дней это впервые могло случиться?

\item
Вершины выпуклого многогранника можно обойти одним циклическим маршрутом.
Докажите, что грани можно правильным образом покрасить в~4 цвета.

\end{problems}


% $date: 2016-06-16
% $timetable:
%   g11r2:
%     2016-06-16:
%       2:

\section*{Числа Каталана}

% $authors:
% - Владимир Алексеевич Брагин
% - Иван Викторович Митрофанов

Последовательность $\{ c_{n} \}$, заданная рекуррентным соотношением
\[
    c_{0} = c_{1} = 1
\, , \quad
    c_{n+1} = c_{0} c_{n} + c_{1} c_{n-1} + \ldots + c_{n} c_{0}
\, , \]
называется последовательностью \emph{чисел Каталана}.
Вот первые члены этой последовательности: $1$, $1$, $2$, $5$, $14$, \ldots

В~следующих задачах задачах нужно доказать (по~индукции или, что лучше,
построив комбинаторную биекцию с~чем-нибудь уже известным), что $c_n$~---
это число

\begin{problems}

\item
\emph{Триангуляций} выпуклого $(n + 2)$-угольника:
разрезаний на~$n$ треугольников непересекающимися диагоналями.

\item
\emph{Неассоциативных произведений} $n + 1$ букв:
способов расставить скобки так, чтобы порядок умножений был однозначно
определен.
Например, для $n = 3$:
\begin{center} \(
    a(b(cd)) \quad (ab)(cd) \quad ((ab)c)d \quad a((bc)d) \quad (a(bc))d
\) \end{center}

\item
\emph{Путей} из~точки $(0, 0)$ в~точку $(n, n)$ по~линиям клетчатой бумаги,
идущих вверх и~вправо, не~поднимающихся выше прямой $y = x$.

\item
\emph{Последовательностей} длины $2n$, в~которых $n$ раз встречается $1$,
$n$ раз встречается $-1$, и~все частичные суммы последовательности
неотрицательны.

\item
\emph{Способов соединить} $2 n$ точек на~окружности $n$ непересекающимися
хордами (из~любой точки выходит одна хорда).

\item
\emph{Плоских корневых двоичных деревьев} с~$n$ вершинами:
у~каждой вершины не~более двух сыновей (правый и левый) и~один предок
(кроме корня, у~которого нет предков).

\item
\emph{Параллеломино} периметра $2n+2$:
пар путей на~клетчатой бумаге с~началом $(0, 0)$ и~концом в~одной и~той~же
точке, идущих только вверх и~вправо и~не~имеющих общих точек, кроме начала
и~конца.

\item
\emph{Путей} из~точки $(0, 0)$ в~точку $(n - 1, n - 1)$, идущих
вправо, вверх или по~диагонали вправо вверх, не~поднимающихся выше
прямой $y = x$ и~таких, что идти по~диагонали можно только вдоль прямой
$y = x$.

\item
\emph{Перестановок чисел} $1$, $2$, \ldots, $n$, у~которых длина каждой
убывающей подпоследовательности не~более $2$.
Например, для $n = 3$:
\begin{center} \(
    123 \quad 213 \quad 132 \quad 312 \quad 231
\) \end{center}

\item
\emph{Способов разбить} все натуральные числа от~$1$ до~$n$ на~несколько
незацепленных групп (нельзя при $a < b < c < d$ отнести $a$ и~$c$ к~одной
группе, а~$b$ и~$d$~--- к~другой).
Например, для $n = 3$:
\begin{center} \(
    123 \quad 12,3 \quad 13,2 \quad 1,23 \quad 1,2,3
\) \end{center}

\item
Докажите явную формулу для чисел этой последовательности:
\[
    c_{n} = \frac{(2 n)!}{n! (n + 1)!}
\, . \]

\end{problems}


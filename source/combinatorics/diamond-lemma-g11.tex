% $date: 2016-06-20
% $timetable:
%   g11r2:
%     2016-06-20:
%       1:
%   g11r1:
%     2016-06-19:
%       2:

\section*{Diamond lemma}

% $authors:
% - Владимир Алексеевич Брагин
% - Иван Викторович Митрофанов

\begin{problems}

\item
В~дереве бесконечное число вершин, а~степень каждой вершины конечная.
Докажите, что можно выбрать простой путь, состоящий из~бесконечного множества
вершин.

\item \claim{Diamond lemma}
Дан ориентированный граф с
\\
\subproblem конечным;
\quad
\subproblem бесконечным
\\
множеством вершин.
Будем называть вершину~$v$ \emph{потомком} вершины~$u$, если существует путь
из~$u$ в~$v$;
если есть ребро из~$u$ в~$v$, то~$v$~--- \emph{ребёнок}~$u$.
Известно, что все пути в~графе конечны (в~частности, нет циклов) и~что
выполнено следующее условие: для любых двух детей любой вершины у~этих детей
существует общий потомок.
Докажите, что у~любой вершины есть лишь один потомок исходящей степени~$0$.

\end{problems}

В~дальнейших задачах этой леммой можно пользоваться без доказательства.

\begin{problems}

\item
На~доске выписаны положительные числа $a_{1}$, $a_{2}$, \ldots, $a_{n}$.
За~ход разрешается взять любые два числа $x$ и~$y$ и~заменить их значением
выражения $x y +x + y$.
Докажите, что финальное число не~зависит от~порядка операций.

\item
В~алфавите имеется $n$ букв и~$n$ соответствующим им антибукв.
Выписано слово этого алфавита.
Каждый квант времени из~слова удаляются случайно выбранные рядом стоящие буква
и~её антибуква (в~любом порядке) до~тех пор, пока не~остаётся несократимое
слово.
Докажите, что результат не~зависит от~хода процесса.

\item
На~доске выписаны натуральные числа $a_{1}$, $a_{2}$, \ldots, $a_{n}$.
За~ход разрешается взять пару чисел, ни~одно из~которых не~делится на~второе,
и~заменить их на~их НОД и~НОК.
\\
\subproblem
Докажите, что как~бы мы ни~действовали, мы не~сможем проворачивать эту операцию
бесконечно много раз.
\\
\subproblem
Докажите, что как~бы мы ни~действовали, мы закончим одним и~тем~же множеством
чисел.

\item
В~ряд стоит $100$ коробок, в~самой левой из~них лежит $100$ спичек.
За~ход разрешается из~любой коробки переложить одну спичку в~соседнюю справа
коробку, при условии, что в~исходной коробке останется не~меньше спичек, чем
получится в~той, куда спичку мы добавили.
Докажите, что результат процесса не~зависит от~порядка операций.

\end{problems}


% $date: 2016-06-21
% $timetable:
%   g11r1:
%     2016-06-21:
%       1:

\section*{Лемма Кёнига и прочее бесконечное}

% $authors:
% - Владимир Алексеевич Брагин
% - Иван Викторович Митрофанов

% $matter[-preamble-package-guard]:
% - .[preamble-package-guard]
% - preamble package: ulem
%   options: [normalem]

\begin{problems}

\item
\subproblem
Существует~ли такое бесконечное множество натуральных чисел, что любые $99$ его
элементов имеют общий делитель, а~любые $100$~--- взаимно простые?
\\
\subproblem
В~бесконечном множестве натуральных чисел у~любого конечного подмножества есть
общий делитель, больший единицы.
Обязательно~ли у~всех чисел есть общий делитель, больший единицы?

\item\emph{Задача отозвана.}
\sout{%
Можно~ли написать в~клетки бесконечного клетчатого листа целые числа так, чтобы
в~каждой строчке и~каждом столбце каждое число встречалось по~одному разу?}

\item \emph{Счётная теорема Рамсея.}
\\
\subproblem
Дано счетное множество людей.
Обязательно~ли среди них есть бесконечно много попарно знакомых или бесконечно
много попарного незнакомых?
\\
\subproblem
Все пары натуральных чисел покрасили в~$50$ цветов.
Обязательно~ли можно выбрать бесконечное множество натуральных чисел, все пары
которого покрашены в~один цвет?
\\
\subproblem
Все тройки натуральных чисел\ldots

\end{problems}

А~теперь пришла пора применять заглавную лемму\ldots

\begin{problems}

\item
Допустим, любую конечную карту можно правильно покрасить в~$4$ цвета.
Выведите из~этого, что бесконечную карту можно покрасить в~$4$ цвета.

\item \emph{Счётная теорема Брукса.}
В~графе со~счетным числом вершин степень каждой вершины не~более 17.
Докажите, что вершины можно правильным образом покрасить в~17 цветов.

\item \emph{Счётная лемма Холла.}
Дано бесконечное множество юношей и~бесконечное множество девушек.
Каждому юноше нравится конечное количество девушек.
Оказалось, что любым $k$~юношам суммарно нравится не~менее $k$~девушек.
Докажите, что можно всем юношам одновременно выбрать пару
(то есть каждому выбрать нравящуюся ему девушку, так чтобы разным юношам
соответствовали разные девушки).

\item
Выведите из~бесконечной теоремы Рамсея конечную.
То есть докажите, что для любого натурального~$k$ и~любого натурального~$n$
существует такое число $R_{k,n}$, что при любой покраске ребер полного графа
на~$R_{k,n}$ вершинах в~$k$ цветов найдется одноцветный полный подграф
на~$n$ вершинах.

\item
В~алфавите конечное количество букв.
\emph{Словом} назовем любую последовательность букв.
Пусть некоторые слова объявлены запретными.
При этом, если у~слова есть запрещенные подслова, то~оно тоже запрещенное.
Оказалось, что любое бесконечное слово запрещенное.
Докажите, что незапрещенных слов конечное число.

\item
В~алфавите конечное количество букв.
Словом назовем любую конечную последовательность букв.
В~языке некоторые слова запрещенные.
При этом, если подслово слова запрещенное, то~само слово тоже запрещенное.
Безумный профессор пишет слова на~доске.
При этом каждую минуту он меняет текущее слово $w$ на~слово $w A w$, где
$A$~--- произвольное слово (вообще говоря, разное для разных действий).
Оказалось, что с~какого~бы слова профессор не~начал свои манипуляции, рано или
поздно он получит на~доске запрещенное слово.
Докажите, что незапрещенных слов конечное количество.

\item
Все натуральные числа покрасили в~несколько цветов.
Докажите,что найдется цвет такой, что для любого натурального~$n$ бесконечно
много чисел этого цвета делится на~$n$.

\item
Натуральные числа покрашены в~несколько цветов.
Докажите, что можно выбрать один из~цветов и~натуральное~$m$ так, что для
любого $N$ существуют $N$ чисел $a_{1} < a_{2} < \ldots < a_{N}$ этого цвета
такие, что $a_{i+1} < a_{i} + m$ для любого $i$.

\end{problems}


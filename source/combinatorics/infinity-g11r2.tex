% $date: 2016-06-22
% $timetable:
%   g11r2:
%     2016-06-22:
%       2:

\section*{Конечное и беконечное}

% $authors:
% - Владимир Алексеевич Брагин
% - Иван Викторович Митрофанов

\begin{problems}

\item
\subproblem
Существует~ли такое бесконечное множество натуральных чисел, что любые 99 его
элементов имеют общий делитель, а~любые 100 взаимно простые?
\\
\subproblem
В~бесконечном множестве натуральных чисел у~любого конечного подмножества есть
общий делитель, больший единицы.
Обязательно~ли у~всех чисел есть общий делитель, больший единицы?

\item
Есть счетное число функций $f_{i} \colon \mathbb{R} \to \mathbb{R}$.
Докажите, что существует функция~$g$ такая, что она больше любой
функции~$f_{i}$ начиная с~некоторого момента~$x_{i}$, возможно, зависящего
от~$i$.

\item
Можно~ли
\\
\subproblem конечным
\quad
\subproblem бесконечным
\\
множеством внутренностей углов с~суммой градусных мер $1^{\circ}$ покрыть
плоскость?

\item
\emph{Плотностью} арифметической прогрессии
$a_{n} = k n + b$, $n \in \mathbb{Z}$, $k > 0$ назовем число $1 / k$.
Можно~ли разбить все целые числа на
\\
\subproblem конечное
\quad
\subproblem бесконечное
\\
число прогрессий с~суммой плотностей меньше единицы?

\item
Есть бесконечная последовательность прямоугольников, площадь $k$-го равна
$k^2$.
Всегда~ли ими можно покрыть плоскость?
(Прямоугольники можно как угодно переносить и~поворачивать).

\item
Каждая точка плоскости с~целыми координатами покрашена в~один из~десяти цветов.
Докажите, что найдется прямоугольник с~одноцветными вершинами.

\item
\subproblem
Дана счетная последовательность действительных чисел, все числа различны.
Докажите, что можно выбрать либо бесконечную последовательность возрастающих
чисел, либо бесконечную последовательность убывающих чисел.
\\
\subproblem
Все пары натуральных чисел покрасили в~два цвета.
Докажите, что можно выбрать бесконечное множество натуральных чисел, все пары
которого покрашены в~один цвет.

\end{problems}

